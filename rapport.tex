\documentclass[a4paper,12pt]{report}
\usepackage[utf8]{inputenc}
\usepackage[T1]{fontenc}
\usepackage{fullpage}
\usepackage{hyperref}
\usepackage{graphicx}
\usepackage[usenames,dvipsnames]{color}
\usepackage{wrapfig}
\usepackage{amsfonts}
\usepackage{pict2e}
\usepackage{listings}
\usepackage{titlesec}
%\usepackage[top=2cm,bottom=2cm,left=2cm,right=2cm]{geometry}
\usepackage{geometry}
\usepackage[final]{pdfpages}
\usepackage[francais]{babel}
\usepackage{caption}
\usepackage{subfig}
\usepackage{amsmath}
\usepackage{tikz}
\usepackage{fancyhdr}
\usepackage[final]{pdfpages}
\usepackage{listings}
\usepackage{color}
\definecolor{dkgreen}{rgb}{0,0.6,0}
\definecolor{gray}{rgb}{0.5,0.5,0.5}
\definecolor{mauve}{rgb}{0.58,0,0.82}
\lstset{frame=tb,
		language=Oz,
 		aboveskip=3mm,
  		belowskip=3mm,
  		showstringspaces=false,
  		columns=flexible,
  		basicstyle={\small\ttfamily},
  		numbers=left,
 		numberstyle=\tiny\color{gray},
  		keywordstyle=\color{mauve},
  		commentstyle=\color{dkgreen},
  		stringstyle=\color{blue},
  		breaklines=true,
  		breakatwhitespace=true
  		tabsize=3
	}
\newcommand{\HRule}{\rule{\linewidth}{0.5mm}}

\setlength{\unitlength}{1mm}

\geometry{hmargin=80pt, vmargin=3cm}


\titleformat{\chapter}
{\bf\fontsize{20.74}{20.74}\selectfont}
{\bf\fontsize{20.74}{20.74}\selectfont \thechapter \hspace{1 cm}}
{0 pt}{}{}
\titlespacing{\chapter}{10 pt}{10 pt}{20 pt}[10 pt]

\newcommand{\MatLab}[1]
{\section{#1}
\lstinputlisting{CodeMatLab/#1.m}}

\newcommand{\note}[1]
{\color{red}#1\color{black}}

\renewcommand{\headrulewidth}{0 pt}
\renewcommand{\footrulewidth}{0 pt}

\begin{document}

\begin{titlepage}

\begin{center}
\textsc{\LARGE  Université Catholique de Louvain}\\[0.5cm]

\textsc{\large Ecole Polytechnique de Louvain\\ 2\up{nd} Bachelier en sciences de l'ingénieur, orientation ingénieur civil\\ 
LFSAB1402 - Informatique 2}\\[1cm]

\textsc{\Large   \\  \vspace{0.8cm}- \today - année 2014-2015 -}\\[0.5cm]
{\large\textbf{Tuteur} : David Sarkozi }
\vspace{0.5cm}
\HRule \\[0.4cm]

{ \huge \bfseries Dj'Oz \\[0.4cm]}

\HRule \\[1.5cm]

\includegraphics[width=(\textwidth /2)]{DJOZ.png}

\vspace{1.0cm}

\emph{Deprez} Damien (28931300), \emph{Mulders} Zélie (80051300)\\

\textbf{Groupe 1235}

 
\end{center}
\end{titlepage}

\vspace*{5cm}
Dans ce rapport, nous allons vous présenter les fonctions que nous avons implémentées dans le cadre du projet du cours d'informatique en deuxième année de bachelier en ingénieur civil. Nous y détaillerons la structure de notre programme, les décisions que nous avons prises, les difficultés rencontrées ainsi que les limites et problèmes de notre implémentation. Nous finirons par l'estimation de la complexité de nos fonctions ainsi que par son extension. Par contre, l'entièreté du code ne se trouve pas dans ce rapport. Vous ne trouverez pas non plus de partie concernant les constructions non-déclaratives car nous n'en n'utilisons pas.

\section*{Structure du programme}
Les deux fonctions principales de notre programme étant relativement indépendantes dans leur implémentation et leurs besoins, nous avons décidé de déclarer les sous-fonctions relatives à chaque fonction à l'intérieur de celle-ci.
Les spécifications de chaque fonction et sous-fonctions se trouvent dans notre programme.

\begin{lstlisting}[frame=single] 
fun{Interprete Partition}
	--Sous-Fonctions--
fun{Mix Interprete Music}
	--Sous-Fonctions--
\end{lstlisting}
\newpage
\subsection*{Interpréter une partition}
Globalement, la structure de la fonction \textit{\textbf{Interprete}} est reprise ci dessous. Nous avons veillé à ce qu'elle soit récursive terminale. Pour cela, nous avons donc créé une fonction \textit{\textbf{InterpreteAux}} reprenant un certain nombre d'arguments supplémentaires à ceux de la fonction \textit{\textbf{Interprete}}. 
\begin{lstlisting}[frame=single] 
fun{Interprete Partition}
	fun{Flatten Partition} %Permet de "deplier" les partitions pour ne plus avoir de listes imbriquees
	fun{NoteToEchantillon Note Duree DemiTons} %Argument note=note etendue ! Demitons=0 si pas de transformations
	fun {DureeTot Partition} %Utile pour la transformation Duree 
	fun {InterpreteAux Partition Note Duree DemiTons Acc}
		case Partition
		[] nil %On renvoie l'accumulateur
		[] Note % decuple en plusieurs cas en fonction du format de la note
		[] muet(TPartition)
		[] etirer(facteur:F TPartition)
		[] duree(seconde:S TPartition)
		[] bourdon(note:NoteB TPartition)
		[] transpose(demitons:DemiTons TPartition)
		[] H|T			
\end{lstlisting}
Notre fonction \textit{\textbf{InterpreteAux}} a donc comme arguments \textit{Partition, Note, Duree, DemiTons} et \textit{Acc} (un accumulateur). Regardons leur utilité plus en détail:
\begin{itemize}
\item \textit{Partition}: Cet argument est la base de la fonction \textit{\textbf{InterpreteAux}} qui a pour but premier d’interpréter une partition. 
\item \textit{Note}: par défaut = nil. Cet argument est utilisé lors des transformations \textbf{muet} et \textbf{bourdon}.
\item \textit{Duree}: par défaut = 1. Cet argument est utilisé pour les transformations \textbf{duree} et \textbf{etirer}.
\item \textit{Demitons}: par défaut = 0. Cet argument est utilisé pour la transformation \textbf{transpose}.
\item \textit{Acc}: l'accumulateur est utilisé pour la récursion terminale. Au fur et à mesure qu'on parcourt la partition, il stocke les notes converties en échantillons.
\end{itemize}
	Comme chaque transformation s'applique sur une partition, nous avons utilisé la récursion pour interpréter les partitions modifiées en changeant les différents arguments de notre fonction auxiliaire \textit{\textbf {InterpreteAux}}. 

\newpage
\subsection*{Mixer la musique}
La base de la structure de notre fonction \textit{\textbf{Mix}} et des sous-fonctions qu'elle utilise est donnée ci-dessous.

\begin{lstlisting}[frame=single] 
fun{Frequence Hauteur}
fun{Add List1 List2}
fun{EchantillonToAudio Echantillon Facteur Acc}
fun{ListEchantillonToAudio Echantillon Facteur}
fun{Clip Up Down MusicClip Facteur}
fun{RepetitionNfois N Music Facteur}
fun{RepetitionDuree Duree Musique Facteur}
fun{Echo Duree Decadence Repetition Music Facteur}
fun{Merge MusicWithIntensity Acc Facteur}
fun{Fondu Ouverture Fermeture Music Facteur}

fun{MixAux Interprete  Music Facteur  Acc}
	of nil then {Flatten Acc}
	[] voix(Voix) then {ListEchantillonToAudio Voix Facteur}
	[] partition(Partition) then {ListEchantillonToAudio {Interprete Partition} Facteur}
	[] wave(File) then {Projet.readFile File}
	[] renverser(MusicR) then{List.reverse {MixAux Interprete MusicR Facteur nil}}
	[] repetition(nombre:N MusicR) then{RepetitionNfois N MusicR Facteur}
	[] repetition(duree:S MusicR) then {RepetitionDuree S MusicR Facteur}
	[] clip(bas:Bas haut:Haut MusiC)  then {Clip Haut Bas MusiC Facteur}
	[] echo(delai:S MusicE) then {Echo S 1.0 1 MusicE Facteur}
	[] echo(delai:S decadence:D MusicE)then{Echo S D 1 MusicE Facteur}
	[] echo(delai:S decadence:D repetition:R MusicE)then {Echo S D R MusicE Facteur}
	[] fondu(ouverture:S1 fermeture:S2 MusicF) then {Fondu S1 S2 MusicF Facteur}
	[] fondu_enchaine(duree:S Music1 Music2) then
	[] couper(debut:S1 fin:S2 Music) then
	[] merge(MusicWithIntensity) then {Merge MusicWithIntensity nil Facteur}
	[] H|T then {MixAux Interprete T Facteur Acc|{MixAux Interprete H Facteur nil}}
\end{lstlisting}
La fonction \textit{\textbf{Mix}} est plus exigeante au niveau des transformations. Pour plus de lisibilité, nous avons fait des sous-fonctions pour chacun des cas pouvant se présenter. 
Par rapport aux arguments de notre fonction \textit{\textbf{MixAux}}, nous avons utilisé l'argument \textit{Facteur} pour pouvoir varier l'intensité de la musique. Nous avons également utilisé un accumulateur pour faire de la récursion terminale.


\section*{Difficultés rencontrées, limitations et problèmes connus}
\begin{itemize}
\item Etant un binôme de deux \textbf{non-musiciens}, le vocabulaire lié à la musique nous a posé quelques difficultés lors de la compréhension du problème, mais également lors de l'implémentation de certaines fonctions.

\item Lors des différents tests de notre programme, nous avons eu des \textbf{parse error} à de nombreuses reprises.  Malheureusement, Mozart n'est pas toujours très explicite dans ses messages d'erreur et nous avons passé beaucoup de temps à les rechercher. 
 
\item Lors des tests, nous n'avions pas encore trouvé comment mettre des \textbf{points d'arrêt} dans Mozart. Ceci complique les recherches des erreurs.

\item Un des problèmes connus de notre fonction est le suivant: à plusieurs reprises, nous appliquons un \textit{\textbf{Reverse}} sur les listes à lire ou créer. Cette opération est tout de même relativement coûteuse pour des listes de grande ampleur.  

 
\end{itemize}


\section*{Complexité des fonctions}
\begin{itemize}
%\item Toutes nos fonctions ont une complexité spatiale en $\vartheta(1)$, car elles sont toutes réursives terminales.
%\item La fonction {Frequence Hauteur} a une complexité spaitale et temporelle en $\vartheta(1)$.

%\item Les fonction {Clip Up Down VecteurAudio}, {EchantillonToAudio Echantillon Facteur Acc} ont une complexité spatiale et temporelle en $\vartheta(n)$. %car elle doit parcourir tout le VecteurAudio et le réécrire. 
%\item La fonction {Add List1 List2} a une complexité spatiale en $\vartheta(n^2)$ et une complexité temporelle en $\vartheta(n)$

\item Fonction \textit{\textbf{Interprete}}: Cette fonction a une complexité spatiale en $\vartheta(1)$, car elle est récursive terminale.
Pour la complexité temporelle, la fonction \textit{\textbf {Interprete}} en elle-même est en  $\vartheta(n)$ et de même pour toutes les sous-fonctions qu'elle utilise. Ceci nous donne une complexité temporelle  générale pour la fonction en  $\vartheta(n)$
\item Fonction \textit{\textbf{Mix}}: Tout comme la fonction \textit{\textbf{Interprete}}, la fonction \textit{\textbf{Mix}} est récursive terminale. Elle a donc une complexité spatiale en  $\vartheta(1)$. Par contre, pour la complexité temporelle, la fonction \textit{\textbf{Mix}} a une complexité temporelle en  $\vartheta(n)$. De plus, comme elle utilise certaines fonctions ayant une complexité en  $\vartheta(n^2)$ (par exemple la fonction \textit{\textbf{Repetition}}) , elle a donc une complexité générale en  $\vartheta(n^2)$
\item Les complexités des différentes sous-fonctions est détaillée en commentaire dans notre programme. 

\end{itemize} 


\section*{Extension}
Nous avons réalisé l'extension {\large \textbf{lissage}} et pour cela nous avons utilisé l'enveloppe ADSR \footnote{\url{http://fr.wikipedia.org/wiki/Enveloppe_sonore}}. Nous avons opté pour une attaque qui part d'une intensité de 0 à 1 sur 2000 valeurs du vecteur audio. Ensuite, nous avons un déclin exponentiel en base deux de 1 à 0.5 d'intensité (sur 8000 valeurs). Puis, nous maintenons l'intensité à 0.5 jusque 4000 valeurs avant la fin du vecteur. A partir de là, nous redescendons de manière linéaire jusqu'à 0.

\end{document}}