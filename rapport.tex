\documentclass[a4paper,12pt]{report}
\usepackage{fullpage}
\usepackage[utf8]{inputenc}
\usepackage[francais]{babel}
\usepackage{listings}
\usepackage{color}
\definecolor{dkgreen}{rgb}{0,0.6,0}
\definecolor{gray}{rgb}{0.5,0.5,0.5}
\definecolor{mauve}{rgb}{0.58,0,0.82}
\lstset{frame=tb,
		language=Oz,
 		aboveskip=3mm,
  		belowskip=3mm,
  		showstringspaces=false,
  		columns=flexible,
  		basicstyle={\small\ttfamily},
  		numbers=left,
 		numberstyle=\tiny\color{gray},
  		keywordstyle=\color{mauve},
  		commentstyle=\color{dkgreen},
  		stringstyle=\color{blue},
  		breaklines=true,
  		breakatwhitespace=true
  		tabsize=3
	}
\begin{document}

Dans ce présent rapport nous allons vous présenter les fonctions que nous avons implémentées dans le cadre du projet du cours d'informatique en deuxième année de bachelier en ingénieur civil. Nous vous détaillerons dans ce rapport la structure de notre programme, les décisions que nous avons prises, les difficultées rencontrées ainsi que les limites et problemes de notre implémentation, nous finirons par estimer la complexité de nos fonctions. Le code dans son entièreté ne se retrouvera pas dans ce rapport.

\section*{Structure du programme}
Les deux fonctions principales de notre programme étant relativement indépendantes dans leur implémentation et leurs besoins, nous avons décidé de déclarer les sous-fonctions relatives à chaque fonction à l'intérieur de celles-ci.
\begin{lstlisting}[frame=single] 
fun{Interprete Partition}
	--Sous-Fonctions--
fun{Mix Interprete Music}
	--Sous-Fonctions--
\end{lstlisting}

\subsection*{Intepréter une partition}
Globalement la structure de la fonction Interprete est reprise ci dessous. Nous avons veillé à ce qu'elle soit récursive terminale, pour cela nous avons donc créé une fonction InterpreteAux reprenant un certains nombre de paramètres en plus que l'originale. 
\begin{lstlisting}[frame=single] 
fun{Interprete Partition}
	fun{Flatten Partition} %Permet de "deplier" les partitions pour ne plus avoir de listes imbriquees
	fun{NoteToEchantillon Note Duree DemiTons} %Argument note=note etendue ! Demitons=0 si pas de transformations
	fun {DureeTot Partition} %Utile pour la transformation Duree 
	fun {InterpreteAux Partition Note Duree DemiTons Acc}
		case Partition
		[] nil %On renvoie l'accumulateur
		[] Note % decuple en plusieurs case en fonction du format de la note
		[] muet(TPartition)
		[] etirer(facteur:F TPartition)
		[] duree(seconde:S TPartition)
		[] bourdon(note:NoteB TPartition)
		[] transpose(demitons:DemiTons TPartition)
		[] H|T			
\end{lstlisting}

Notre fonction InterpreteAux a donc comme paramètres Partition, Note, Duree, DemiTons et Acc (un accumulateur). Regardons leur utilité plus en détail:
\begin{itemize}
\item Partition: rien de surprenant ici puisque la fonction interprete à pour but premier interpreter une partition. 
\item Note: par défaut = nil, cet argument est utile lors des transformations muet et bourdon. En effet, lorsqu'une de ses transformations est appelée, Note est remplacé respectivement par un silence ou la note à "bourdonner".  
\item Duree: pour une note n'ayant subit aucune transformation, la durée est d'une seconde. Ce paramètre est donc utilisé pour le cas ou la transformation duree et etirer.
\item Demitons: une fois de plus cet argument est initialisé à 0 et ne change que dans le cas d'une transformation transpose.
\item Acc: l'accumulateur comme on peut s'en douter est utile pour la récursion terminale, car au fur et à mesure qu'on parcourt la partition il stocke les notes converties en échantillons. 
\end{itemize}


Comme chaque transformation s'applique sur une partition, nous avons utilisé la récursion pour interpreter les partitions modifiées en changeant les différents paramètres de notre fonction auxiliaire.


\subsection*{Mixer la musique}
Le gros de la structure de notre fonction Mix et des sous fonctions qu'elle utilise est donnée ci-dessous.

\begin{lstlisting}[frame=single] 
fun{Frequence Hauteur}
fun{EchantillonToVecteurAudio Echantillon}
fun{Clip Up Down VecteurAudio}
fun{RepetitionNfois N Music}
fun{RepetitionDuree Duree Musique}
fun{Echo Delai Music}
fun{EchoDecadence Delai Decadence Music}

fun{Mix Interprete Music}
	fun{MixAux Interprete  Music  Acc}
	of partition(P) then {Mix Interprete voix({Interprete P})}
	[] voix(H1|T1) then {EchantillonToVecteurAudio H1}
	[] wave(filename) then {Projet.readFile filename}
	[] merge(musiqueIntensifiee)
	[] renverser(musique) 
	[] repetition(nombre:nat musique) then {RepetitionNfois repetition.nombre repetition.1}
	[] repetition(duree:sec musique) then {RepetitionDuree repetition.duree}
	[] clip(bas:float haut:float musique) then {Clip clip.haut clip.bas {Mix Inteprete clip.1}}
	[] echo(delai:sec musique) then  {Echo echo.delai echo.1}
	[] echo(delai:sec decadence:float musique) then {EchoDecadence echo.delai echo.decadence echo.1}
	[] echo(delai:sec decadence:float repetition:entier musique)
	[] fondu(ouverture:sec fermeture:sec musique)
	[] fondu_enchaine(duree:sec musique1 musique2)

\end{lstlisting}

Puisque dans le cas de la fonction Mix, nous ne traitons plus des partitions, mais de la musique qui ne s'exprime que sous la forme d'une simple liste sans imbrication, il n'est plus nécessaire de faire une fonction auxiliaire avec de nombreux arguments. Pour rendre notre fonction récursive terminale, nous avons donc simplement fait une fonction auxiliaire avec un accumulateur. 


\section*{Difficultés rencontrées, limitations et problèmes connus}
\begin{itemize}
\item Etant un binome de deux \textbf{non-musiciens}, le vocabulaire lié à la musique nous a posé quelques problèmes lors de la compréhension du problème, mais également lors de l'implémentation de la fonction NoteToEchantillon. Celle-ci consistait à déterminer le nombre de demitons séparant la note courante de la note a4 de référence. Nous avons du réimplémenter cette fonction à plusieurs reprises après s'etre rendu compte que nous avions mal compris la logique des demitons entre chaque note et chaque octave. Les documentations Wikipedia ont du etre relues plusieurs fois afin de réellement comprendre ces différents concept.

\item Lors des différents tests de notre programme, nous avons eu des \textbf{parse error} à de nombreuses reprises, malheureusement Mozart n'est pas toujours très explicite dans ses erreurs et nous avons passé beaucoup de temps à les dénicher. 
 
\item Nous n'arretons pas de retourner nos listes.
\end{itemize}


\section*{Complexité des fonctions}



\end{document}}